\documentclass{my_cv_bis}
\hypersetup{
	}
 
\begin{document}
\heading{Antoine}{de Roquemaurel}{Développeur}
 {
	\contact{
		7 impasse André Lartigue\\
		Bât. F2, Appt 41\\
		31500 Toulouse
	}{
		\mobile{06~84~33~52~93}\\
		\email{\texttt{\href{mailto:antoine.roquemaurel@gmail.com}{\color{white}{antoine.roquemaurel@gmail.com}}}}\\
		\texttt{\hspace{-20px}\Mundus~\href{https://aroquemaurel.github.io}{\color{white}{https://aroquemaurel.github.io}}}
	}{	
		Né le 08/11/1991 -- Permis B \\
		~
	}
}{Photo_mini}

%	\section{Références}
%		\reference{Guillaume}{Cabanac}{Enseignant--Chercheur}{IRIT}{Toulouse}{05.61.55.72.73}
%
%		\reference{James}{Ashton}{Responsable du système d'information}{IEP Toulouse}{Toulouse}{05.61.11.04.58}


 
 \vspace{-5mm}
\section{Expériences professionnelles}
\cventry{\proleft{2013~--~Aujourd'hui}{Freelance}{C++, Qt5}}
	{Conception d’une application de gestion de bilans}
	{Autoentrepreneur}{Toulouse}
	{
	Développement d'une application de réalisation de différents bilans médicaux à destination des kinésithérapeutes 
	de l'Association. À terme l'application permettra l'acquisition et l'interprétation des données.
	}
	{\course{Site web de l'association : \url{http://www.afmck.com}}}
	\cventry{\proleft{Avril~--~Juillet~2014}{Stage}{Java, Antlr4}}
	{Participation au développement d'une plateforme de tests automatisés}
	{Continental Automotive}{Toulouse}
	{
	La plateforme à pour but de tester la bonne intégration d'un plugin présent dans un logiciel de calculateur de contrôle moteur en exécutant
	automatiquement les tests spécifiés, suivi d'un rapport détaillé.
	}
	{}

	\cventry{\proleft{Juillet 2012}{CDD}{PHP5, Mootools}}
	{ Participation au développement d'une application web}
	{Memobox / 2G Technologies}{Toulouse}
	{Développement d'un nouveau système de traduction basé sur un moteur de traduction PHP et un client lourd
	permettant la traduction et la recherche de texte dans une base de données.  } {}

	\cventry{\proleft{Avril~--~Juin 2012}{Stage}{PHP5, Adobe AIR}}
	{ Refonte d’un système de traduction }
	{Memobox / 2G Technologies}{Toulouse}
	{Développement d'un nouveau système de traduction basé sur un moteur de traduction PHP et un client lourd
	permettant la traduction et la recherche de texte dans une base de données.  } {}

	\cventry{\proleft{2009~--~Aujourd'hui}{Freelance}{PHP5, Symfony2}}
	{Élaboration d’un site Internet}
	{Yverdon-les-bains}{Suisse}
	{Développement du site d’un artisan verrier, destiné à exposer ses pièces. L'artisan peut administrer le site
	seul: ajouter de nouvelles photo, ajouter des événements, changer le contenu du site,\ldots}
	{}

	\subsection{Projets personnels}
	\cventry{Décembre 2010, 2011 et 2012\\{\vspace{0.5mm}\small\color{gray}~ }}
	{Participation à la nuit de l'informatique}
	{Université Toulouse III -- Paul Sabatier}{Toulouse}
	{ Concours national d’informatique réunissant des étudiants de toute la France. Le concept est de développer une application web en une nuit.} 
	{\course{\url{http://nuitdelinfo.com}}}
	\vspace{-20px}
\section{Formation}
	\cventry{2014~--~2015}
	{Master 1 Développement Logiciel}
	{Université Toulouse III -- Paul Sabatier}{Toulouse}{}{}
	\cventry{2012~--~2014}
	{Licence Informatique}
	{Université Toulouse III -- Paul Sabatier}{Toulouse}{Parcours Ingénierie des Systèmes Informatiques}{}
	\cventry{2010~--~2012}
	{DUT Informatique}
	{IUT 'A' Paul Sabatier}{Toulouse}
	{Options Méthodes Agiles, Interface graphique en Java, Création d'entreprise}
	{}
	\cventry{2008~--~2010}
	{Baccalauréat Sciences et Techniques de l'Industriel}
	{Lycée Saint Joseph}{Toulouse}
	{Génie Électrotechnique}{}
	\vspace{-20px}
\section{Compétences}
%	\begin{minipage}[t]{0.45\textwidth}
		\subsection{Informatique}
		\cvline{\textbf{Programmation}}{C, C++, Java, PHP, JavaScript}
		\cvline{\textbf{Frameworks}}{Qt, Symfony, Mootools, Bootstrap, Adobe AIR}\\
		\cvline{\textbf{Versionnement}}{Git, Subversion}\\
		\cvline{\textbf{SGBD}}{MySQL, Oracle}\\
		\cvline{\textbf{Système}}{Linux, UNIX, Windows}\\
		\cvline{\textbf{Bureautique}}{\LaTeX, Suite Microsoft Office, Suite Apache Openoffice}\\
%	\end{minipage}
%	\hfill
		\subsection{Langues}
		\cvline{\textbf{Anglais}}{Technique lu et écrit}
		\cvline{\textbf{Français}}{Langue maternelle }
	\vspace{-10px}
\section{Loisirs et activités}
\cvline{\textbf{Associatif}}{
Membre des organisateurs de la \textbf{nuit de l'info} toulousaine (2013, 2014)\newline 
Animateur bénévole Scouts et Guides de France. Enfants de 6 à 11 ans (2009 -- 2012)
}
		\cvline{\textbf{Culture}}{Lecture de science fiction (Asimov) et de fantastique (Tolkien)}

\end{document} 
