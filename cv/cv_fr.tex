\documentclass{my_cv_bis}
\hypersetup{
	}
 
\begin{document}
\heading{Antoine}{de Roquemaurel}{Développeur}
 {
	\contact{
		7 impasse André Lartigue\\
		Bât. F2, Appt 41\\
		31500 Toulouse
	}{
		\mobile{06~84~33~52~93}\\
		\email{\texttt{\href{mailto:antoine.roquemaurel@gmail.com}{\color{white}{antoine.roquemaurel@gmail.com}}}}\\
		\texttt{\hspace{-20px}\Mundus~\href{https://antoinederoquemaurel.github.io}{\color{white}{https://antoinederoquemaurel.github.io}}}
	}{	
		Né le 08/11/1991 -- Permis B \\
		~
	}
}{Photo_mini}

 \vspace{-5mm}
\section{Expériences professionnelles}
\cventry{\proleft{2013~--~Aujourd'hui}{Freelance}{C++, Qt5, PHP, Symfony2}}
	{Développeur Freelance}
	{Autoentrepreneur}{Toulouse}
	{
	Développement de logiciels ou sites web, tel qu'un logiciel de bilans kinésithérapeutes, le site web d'une association (\url{afmck.fr})
	ou celui d'un artisan verrier(\url{valeriederoquemaurel.com})
	}
	{\course{Plus d'informations sur \url{https://antoinederoquemaurel.github.io/\#portfolio}}}

	\cventry{\proleft{Mai~--~Aout~2015}{Stage de M1}{Java, Antlr4}}
	{Participation au développement d'une plateforme de tests automatisés}
	{Continental Automotive}{Toulouse}
	{
	Suite du projet commencée lors du stage précédent : ajout de nouvelles fonctionnalités en vue de l'utilisation
	de la plateforme de tests sur des projets destinés à être déployés en production. 
	}
	{}
	\cventry{\proleft{Avril~--~Juillet~2014}{Stage de Licence}{Java, Antlr4}}
	{Participation au développement d'une plateforme de tests automatisés}
	{Continental Automotive}{Toulouse}
	{
	La plateforme à pour but de tester la bonne intégration d'un plugin présent dans un logiciel de calculateur de contrôle moteur en exécutant
	automatiquement les tests spécifiés, suivi d'un rapport détaillé.
	}
	{}

	\cventry{\proleft{Juillet 2012}{CDD}{PHP5, Mootools}}
	{ Membre d'une équipe de développeurs Web}
	{Memobox / 2G Technologies}{Toulouse}
	{Développement d'une application SaaS exploitant des données Télécom. Le contrat avait pour but de développer un système de
	communication simple avec des fournisseurs.} {}

	\cventry{\proleft{Avril~--~Juin 2012}{Stage de DUT}{PHP5, Adobe AIR}}
	{ Refonte d’un système de traduction }
	{Memobox / 2G Technologies}{Toulouse}
	{Développement d'un nouveau système de traduction basé sur un moteur de traduction PHP et un client lourd
	permettant la traduction et la recherche de texte dans une base de données.  } {}
	\subsection{Projets personnels}
	\cventry{Décembre 2010, 2011 et 2012\\{\vspace{0.5mm}\small\color{gray}~ }}
	{Participation à la nuit de l'informatique}
	{Université Toulouse III -- Paul Sabatier}{Toulouse}
	{ Concours national d’informatique réunissant des étudiants de toute la France. Le concept est de développer une application web en une nuit.} 
	{\course{\url{http://nuitdelinfo.com}}}
	\vspace{-20px}
\section{Formation}
	\cventry{2014~--~2015}
	{Master 1 Développement Logiciel}
	{Université Toulouse III -- Paul Sabatier}{Toulouse}{}{}
	\cventry{2012~--~2014}
	{Licence Informatique}
	{Université Toulouse III -- Paul Sabatier}{Toulouse}{Parcours Ingénierie des Systèmes Informatiques}{}
	\cventry{2010~--~2012}
	{DUT Informatique}
	{IUT 'A' Paul Sabatier}{Toulouse}
	{Options Méthodes Agiles, Interface graphique en Java, Création d'entreprise}
	{}
	\cventry{2008~--~2010}
	{Baccalauréat Sciences et Techniques de l'Industriel}
	{Lycée Saint Joseph}{Toulouse}
	{Génie Électrotechnique}{}
	\vspace{-20px}
\section{Compétences}
%	\begin{minipage}[t]{0.45\textwidth}
		\subsection{Informatique}
		\cvline{\textbf{Programmation}}{C++, Java, C, PHP, Python, Bash, Groovy, JavaScript}
		\cvline{\textbf{Frameworks}}{Qt, Symfony2, Django, Grails, JQuery, Mootools, Bootstrap, Adobe AIR}\\
		\cvline{\textbf{Versionnement}}{Git, Subversion}\\
		\cvline{\textbf{SGBD}}{MySQL, SQLite, Oracle}\\
		\cvline{\textbf{Système}}{Linux, UNIX, Windows}\\
		\cvline{\textbf{Bureautique}}{\LaTeX, Suite Microsoft Office, Suite Apache Openoffice}\\
%	\end{minipage}
%	\hfill
		\subsection{Langues}
		\cvline{\textbf{Anglais}}{Technique lu et écrit}
		\cvline{\textbf{Français}}{Langue maternelle }
	\vspace{-10px}
\section{Loisirs et activités}
\cvline{\textbf{Associatif}}{
Membre des organisateurs de la \textbf{nuit de l'info} toulousaine (2013, 2014)\newline 
Animateur bénévole Scouts et Guides de France. Enfants de 6 à 11 ans (2009 -- 2012)
}
		\cvline{\textbf{Culture}}{Lecture de science fiction (Asimov) et de fantastique (Tolkien)}

\end{document} 
