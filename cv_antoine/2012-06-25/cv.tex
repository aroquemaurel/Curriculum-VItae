%% Exemple de CV en LaTeX. 
%% Émilia Robin (Emilia.Robin@ens.fr)
%% janvier 2000

\documentclass[a4paper,11pt]{article} % fonte 11 points, papier a4

\usepackage{lmodern}
\usepackage{xcolor}
\usepackage[utf8]{inputenc}
\usepackage[T1]{fontenc}
\usepackage[francais]{babel}
\usepackage[top=1.7cm, bottom=1.7cm, left=1.7cm, right=1.7cm]{geometry}
%\usepackage[frenchb]{babel}
%\usepackage{layout}
%\usepackage{setspace}
%\usepackage{soul}
%\usepackage{ulem}
%\usepackage{eurosym}
%\usepackage{bookman}
%\usepackage{charter}
%\usepackage{newcent}
%\usepackage{lmodern}
%\usepackage{mathpazo}
%\usepackage{mathptmx}
%\usepackage{verbatim}
%\usepackage{moreverb}
%\usepackage{wrapfig}
%\usepackage{amsmath}
%\usepackage{mathrsfs}
%\usepackage{asmthm}
%\usepackage{makeidx}
\usepackage{listings}
\usepackage{fancyhdr}
\usepackage{multido}
%\usepackage{url}

% La page
%#########
\pagestyle{empty}               % On ne numérote pas les pages
\usepackage{vmargin}            % redéfinir les marges
\setmarginsrb{3cm}{3cm}{3cm}{3cm}{0cm}{0cm}{0cm}{0cm}

% Marge gauche, haute, droite, basse; espace entre la marge et le texte à
% gauche, en  haut, à droite, en bas

% Diverses nouvelles commandes
%#############################

% Pour laisser de l'espace entre les lignes du tableau
\newcommand\espace{\vrule height 20pt width 0pt}

% Pour mes grands titres
\newcommand{\titre}[1]{%
	\begin{center}
	\bigskip
	\rule{\textwidth}{1pt}
	\par\vspace{0.1cm}
        \textbf{\LARGE #1}
	\par\rule{\textwidth}{1pt}
	\end{center}
	\bigskip
	}

% Début du document
%###################
\begin{document}
%###################

\begin{center}
\par\textbf{\huge Curriculum Vitae}
\end{center}

\vspace{1.5cm}

\begin{flushleft}
Antoine de Roquemaurel\\
23, avenue André Bousquairol\\
31\,400 Toulouse\\

\medskip
\begin{flushright}
Tél.: 09 54 48 71 94
Tél.: 06 84 33 52 94

E-mail: antoinederoquemaurel@yahoo.fr


\end{flushright}

\begin{flushleft}
Nationalité française \\
Née le 08 novembre 1991.
\end{flushleft}

\end{flushleft}

\titre{Formation}
%#############

\begin{tabular}{c@{ :  }p{0.8\textwidth}}


\textbf{2008-2010} & Baccalauréat Sciences et Techniques de l'Industriel Génie Electrotechnique \\

\espace
\textbf{2011} & Diplome universitaire de Technologie Informatique 
\espace
\end{tabular}
\subsubsection*{Compétences linguistiques}
	Anglais \\
\subsubsection*{Compétences techniques}
	Programmation: C, C++\\
	Web: XHTML, CSS, PHP\\
	Système: Unix, Windows\\
	Base de données: MySQL
\titre{Expérience}
%#########################


\begin{itemize} 

\item \textbf{Espagnol}, couramment (cours d'été du Colegio de España à
Salamanque en 1993, licence d'espagnol en 1995, DELE en 1996)

\medskip
\item \textbf{Anglais}  (séjours réguliers en Angleterre  pendant cinq ans)

\medskip
\item \textbf{Allemand}, lu

\medskip
\item \textbf{Notions de russe} (cours de l'INALCO, cours d'été de
l'Université d'État de Saint-Pétersbourg en 1997)

\end{itemize}

\titre{Centre d'intérets}
%###################

\begin{itemize}

\item \textbf{Informatique}: j'ai créé un système de tutorat informatique dans
le cadre de l'ENS, pour aider les élèves débutants (organisation de stages,
rédaction d'un mensuel, entretien d'un site Web: 

Je travaille sous Solaris ou Linux, et utilise couramment LaTeX et le langage
HTML. Dans le cadre des stages, j'ai assuré environ 150~heures d'enseignement.

\item \textbf{Chantiers de jeunes} avec l'association «Jeunesse et
Reconstruction» (1994, 1995, animatrice en 1996, 1998 et 1999).  Dans ce
cadre, j'ai eu en charge  des groupes de 10 à 15 personnes de nationalités et
de langues différentes, pendant trois semaines, ainsi que la gestion des
rapports entre l'association et la commune dans le cadre du chantier.

\end{itemize}

\end{document}
