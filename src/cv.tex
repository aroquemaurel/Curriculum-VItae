\documentclass{cv_theme} 

\usepackage{wrapfig}
\hypersetup{
	}
 
\begin{document}
\heading{Antoine}{de Roquemaurel}{Développeur logiciel Junior}
 {
	\contact{
		~\\~\\
		92500 Rueil-Malmaison\\
		France
	}{
		\mobile{Préférence de contact par mail}\\
		\email{\texttt{\href{mailto:antoine.job@roquemaurel.pro}{\color{white}{antoine.job@roquemaurel.pro}}}}\\
		\texttt{\hspace{-50px}\Mundus~\href{https://antoinederoquemaurel.github.io}{\color{white}{https://antoinederoquemaurel.github.io}}}
	}{	
		Né le 08/11/1991 -- Permis B \\
		~
	}
}{Photo_mini}

\section{Expériences professionnelles}

\cventry{\proleft{2016~--~Aujourd'hui}{Prestataire Extia}{2 ans\\En cours}}{Ingénieur développeur consultant}{TECH'Advantage}{Rueil-Malmaison, France}
{
	\vspace{-0.4cm}
	\begin{wrapfigure}{r}{2cm}
		\vspace{0.2cm}
		\centering
		\includegraphics[width=1.5cm]{images/logos/extia.png}\vspace{0.5cm}
		\includegraphics[width=2.5cm]{images/logos/tech_advantage.png}
	\end{wrapfigure}	
	 Développement d'un plugin Eclipse RCP ayant pour but la simulation d'écoulements de fluides dans le cadre d'une suite logicielle pour l'exploitation pétrolière.\\~\\
	Responsable technique du projet depuis Mars 2018 :\\
	-- Développement, conception, architecture\\
	-- Relecture de code, intégration\\
	-- Suivi des développeurs\\
	-- Analyse des besoins et des exigences\\
	-- Rédaction des spécifications\\
	-- Chiffrage
}
{
	Technologies : Java, Eclipse RCP, Linux, Git, Jenkins, Subversion 
	\vspace{15px}
	}
\cventry{\proleft{2015~--~2016}{Contrat de professionnalisation}{1 an}}{Développeur d'outils métiers}{Continental
Automotive}{Toulouse, France}
{
	\vspace{-0.4cm}
	\begin{wrapfigure}{r}{2cm}
		\vspace{-1cm}
		\includegraphics[width=2cm]{images/logos/continental.png}
	\end{wrapfigure}	
-- Analyse des besoins et des exigences\\
-- Établissement d'un \textit{workflow} Git (\textit{Feature branch})\\
-- Support et formation des utilisateurs\\
-- Conception, développement 
}
{
	Technologies : UML, Java, Python, \LaTeX, Git 
	\vspace{15px}
}
\cventry{\proleft{2013~--~Aujourd'hui}{Freelance}{5 ans\\En cours}}
	{Developpeur Freelance}
	{Autoentrepreneur}{Toulouse}
	{
	\vspace{-0.4cm}
	\begin{wrapfigure}{r}{1.8cm}
		\vspace{-0.4cm}
		\includegraphics[width=1.8cm]{images/logos/afmck.jpg}
	\end{wrapfigure}	
		-- Logiciel de bilans kinésithérapeutes\newline
		-- Site web d'une association (\url{afmck.fr})\newline
		-- Site web d'un artisan verrier (\url{https://valeriederoquemaurel.com})\newline
		\course{Plus d'informations sur \url{https://antoine.roquemaurel.pro/\#projects}}
	}
	{
	Technologies : C++, Qt5, PHP, Symfony2, \LaTeX{}
	\vspace{15px}
	}
	\vspace{15px}
	\cventry{\proleft{\mbox{Mai~--~Août~2015}}{Stage M1}{4 mois}}
	{Développement d'une plateforme de tests automatisés}
	{Continental Automotive}{Toulouse, France}
	{
	\begin{minipage}{0.85\textwidth}
	\begin{wrapfigure}{r}{2cm}
		\vspace{-1cm}
		\includegraphics[width=2cm]{images/logos/continental.png}
	\end{wrapfigure}	
	Suite du projet commencé lors du stage précédent : Analyse, conception et développement de nouvelles fonctionnalités en vue de l'utilisation
	de la plateforme sur des projets destinés à être déployés en production. 
	\end{minipage}
	}
	{Technologies: UML, Java, Python}
	\vspace{15px}
	\cventry{\proleft{Avril~--~Juillet~2014}{Stage L3}{3 mois}}
	{Conception d'une plateforme de tests automatisés}
	{Continental Automotive}{Toulouse, France}
	{
	\begin{minipage}{0.85\textwidth}
	\begin{wrapfigure}{r}{2cm}
		\vspace{-1cm}
		\includegraphics[width=2cm]{images/logos/continental.png}
	\end{wrapfigure}	
	Co-élaboration de l'architecture d'une plateforme de tests ayant pour but de valider la bonne intégration d'un plugin présent dans un logiciel de calculateur de contrôle moteur. Cette
	plateforme a pour but d'exécuter des tests automatiquement et en générer des rapports détaillés.
	\end{minipage}
	}
	{
	Technologies : UML, Java, Antlr4
	}
	\vspace{15px}
	\cventry{\proleft{Juillet 2012}{CDD}{1 mois}}
	{Analyse programmeur Web}
	{Memobox / 2G Technologies}{Toulouse, France}
	{%
	\begin{minipage}{0.85\textwidth}
	\begin{wrapfigure}{r}{2cm}
		\vspace{-0.5cm}
		\includegraphics[width=2cm]{images/logos/memobox.jpg}
	\end{wrapfigure}	
	Membre d'une équipe de 3 personnes développant une application SaaS (\textit{Software as a Service}) exploitant les données de télécommunications. 
	Développement d'un module de communication simple avec des fournisseurs.
	\end{minipage}
	} {Technologies: PHP5, Mootools}
	\vspace{15px}
	\cventry{\proleft{Avril~--~Juin 2012}{Stage DUT}{3 mois}}
	{ Refonte d’un système de traduction }
	{Memobox / 2G Technologies}{Toulouse, France}
	{
	\begin{minipage}{0.85\textwidth}
	\begin{wrapfigure}{r}{2cm}
		\vspace{-0.4cm}
		\includegraphics[width=2cm]{images/logos/memobox.jpg}
	\end{wrapfigure}	
	Développement d'un nouveau système de traduction basé sur un moteur de traduction PHP et un client lourd
	permettant la traduction et la recherche de texte dans une base de données.
	\end{minipage}
	} {Technologies: PHP5, Adobe AIR}
	
		\newpage
		\newgeometry{left=1cm,top=1cm,right=2cm, bottom=1cm,noheadfoot}
	\subsection{Projets personnels}
	\cventry{Avril~--~Mai~2015\\{\vspace{0.5mm}\small\color{gray}~ }}
	{Conception d'un logiciel de facturation}
	{AutoEntrepreneur}{Toulouse, France}
	{Logiciel de facturation et devis et gestion des clients. Développé en C++ (framework << Qt >>)
	} 
	{Technologies : C++, Qt5, SQLite, \LaTeX}

	\cventry{Décembre 2010, 2011 et 2012\\{\vspace{0.5mm}\small\color{gray}~ }}
	{Compétiteur à la \textit{nuit de l'info}}
	{Université Toulouse III -- Paul Sabatier}{Toulouse, France}
	{ Concours national d’informatique réunissant des étudiants de toute la France. Le concept est de développer une application web en une nuit.} 
	{\course{\url{http://nuitdelinfo.com}}\vspace{10px}}

		\section{Compétences}
			\subsection{Informatique}
			\cvline{\textbf{Programmation}}{Java, C++, C, Python, PHP, Bash, Groovy}
			\cvline{\textbf{Frameworks}}{Qt5, Django, Symfony2, JQuery, Bootstrap}\\
			\cvline{\textbf{Conception}}{UML2, SysML, Merise}\\
			\cvline{\textbf{Versionnement}}{Git, Subversion}\\
			\cvline{\textbf{SGBD}}{MySQL, SQLite, Oracle, H2}\\
			\cvline{\textbf{Systèmes}}{Linux, UNIX, Windows}\\

		\subsection{Langues}
			\cvline{\textbf{Anglais}}{Technique lu et écrit}
			\cvline{\textbf{Français}}{Langue maternelle}
			\vspace{10px}

	\section{Formation}
	\cventry{2014~--~2016}
	{\textnormal{Master Développement Logiciel}}
	{Université Toulouse III -- Paul Sabatier}{France}{\textit{Systèmes distribués}
	\newline~-- Versionnement\newline -- Conception (UML, SysML)\newline -- Architecture\newline -- Gestion de projets (Methodes
	traditionnelles, approches agiles)}{}
	\cventry{2012~--~2014}
	{\textnormal{Licence Informatique}}
	{Université Toulouse III -- Paul Sabatier}{France}{\textit{Ingénierie des Systèmes Informatiques}\newline 
	-- Programmation (C, C++, Java)\newline -- Réseaux \newline -- Base de données}{}
	\cventry{2010~--~2012}
	{\textnormal{DUT Informatique}}
	{IUT 'A' Toulouse}{Toulouse, France}
	{~--~Programmation\newline-- Administration système\newline-- Base de données avancée}
	{}
	\cventry{2008~--~2010}
	{\textnormal{Baccalauréat Sciences et Techniques de l'Industriel}}
	{Saint Joseph}{Toulouse, France}
	{\textit{Génie Electrotechnique}}{}
	\vspace{10px}
\section{Loisirs}
\cvline{\textbf{Associatif}}{
Adhérent à Zeste de Savoir site web de transmission de connaissances (\texttt{\href{https://zestedesavoir.com}{zestedesavoir.com}})\newline
						Membre des organisateurs de la \textit{\textbf{nuit de l'info}} (2013~--~2017)\newline 
						Bénévole Scouts et Guides de France. Encadrement d'enfants de 6 à 11 ans (2009 -- 2012)
}

\cvline{\textbf{Divertissements}}{
						Cinéma et séries (\texttt{\href{https://senscritique.com/antoinederoquemaurel}{senscritique.com/antoinederoquemaurel}})\newline~
						Photographie de paysages, portraits, objets (\texttt{\href{https://flickr.com/photos/aroquemaurel/}{flickr.com/photos/aroquemaurel})}\newline
						Lecture de science fiction et fantastique \newline~
						Voyages (Californie, Portugal, Espagne, Italie, Angleterre)
}

\cvline{\textbf{Administration système}}{
						Utilisation d'un Raspberry Pi comme media-center domestique \newline
						Serveur de développement personnel (Git, Gitlab, Jenkins, SonarQube, Redmine)
}

%\section{References}
%\cvline{\textbf{Alain Fernandez \textnormal{(Ref1)}}}{
%Confirmed embedded developer, \textit{Continental Automotive}, co-worker in tool for automatic tests\newline
%\Letter~\url{alain.fernandez@continental-corporation.com}\newline
%\Telefon~+335 61 19 54 91
%\newline
%}
%\cvline{\textbf{Jacky Otero \textnormal{(Ref2)}}}{
%Physiotherapist, \textit{Association Française McKenzie}, customer in self-employee experience\newline
%\Letter~\url{jacky.otero@wanadoo.fr}\newline
%\Mobilefone~+336 86 22 45 44 
%}
\end{document} 
